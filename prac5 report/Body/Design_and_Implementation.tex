\section{Design and Implementation }

\subsection{Hardware ans Software}
This was run on a MacBook Pro computer using Iverilog. Additionally, gtkwave was used to monitor the wires.

\subsection{TSC design overview}

The TSC (Trigger Surround Cache) has a 3 bit state register, a 32-bit timer,
a 32-bit TRIGGER\_TM, and an internal ring buffer.
It is connected to the ADC (Analog-to-Digital Converter)
via a request (REQ), ready (RDY), and data (DAT) lines.
Additionally, the TSC can communicate with other devices using triggered (TRD), Send Buffer (SBF),
serial data (SD), and completed data (CD) registers and wires.


\begin{figure}[H]
      \centering
      \includegraphics[width=0.8\columnwidth]{Figures/block_diagram_of_TSC}
      \caption{Block diagram of the TSC}
      \label{fig:block diagram of TSC}
\end{figure}

There is also an accompanying TSC\_tb test bench which is used to initiate and test the TSC module.

\subsection{CLock (clk)}
A 250 MHz clock signal is set up on clk wire in the TS\_tb test bench.

\subsubsection{State register}
The state register is a 3-bit register that has the folowing states:
\begin{itemize}
      \item 000 Stop :
            State when the machine is powered on and has not been reset yet.
      \item 001 ready :
            State that is enter on the reset pin rising edge and it waits for the start pin rising edge.
      \item 010 Running :
            State that is entered from ready or Idle state once the start pin rising edge is pulled hight.
            It incrementing the timer and wright adc values to the ring buffer.
      \item 011 Triggered :
            State entered when the value read from the ADC is
            greater than the predetermined trigger value (TRIGVL).
            It capturing the next 16 values.
      \item 100 IDLE ;
            This state is entered when a trigger event has occurred
            and the TSC is waiting for the start pin or SPF pins rising edge.
      \item 101 SENDING :
            This state is entered from the IDLE state when the SFB gose high. It indicates that data is being sent on the SD line.
\end{itemize}


\subsubsection{Timer}
The timer is incremented on the rising edge of the clock. When
a trigger even occurs the timer is save in the TRIGTM register which is outputted to the test bench.
The timer is reset if a transition into a running state occurs.
To calculate the time store in the TRIGTM register the timer is multiplied by the clock period (4 ps).



\subsubsection{Ring Buffer}
The ring buffer is used to store the values read by the ADC.
It is made up of 32 8-bit registers stored in an array called ring\_buffer.
The tail pointer is named write\_prt and is Initial set to 5'b11111.
and the head pointer is named read\_ptr and is initial set to 5'b00000.
The 5 bit format for the head and tail index is used to induce role offer at value 32 (32 just becomes 0).
To add a new value to the ring buffer the write\_ptr is incremented and then the
value is stored in the ring buffer at the write\_ptr index then read\_ptr is incremented.
To read a value from the ring buffer the value at the read\_ptr index is read and the read\_ptr is incremented.
this proses is repeated until the read\_prt = wright\_ptr. Indicates that all the values have been read.

\subsubsection{How the TSC intervacec with the ADC}
The ADC is initialized in the TSC module. this is creates the adc\_request, adc\_request,  adc\_ready, and DAT wires.
The adc\_request line is to the main reset line this mean that the ADC is reset when the TSC\_td module pulls the reset pin hight.
once the ADC is reset the adc\_ready line is pulled hight to indicate that the ADC is ready to send data.
When the TSC module detects the adc\_ready line is hight and the TSC is in the running state.
The TSC will pull the adc\_request line hight on  the posedge of the clk line for 1 ps to request data from the ADC. This can be seen in the two code section below.

\begin{lstlisting}[language=Verilog, caption={Code for storing data and moving pointers i the posedge adc\_ready}]
      always @(posedge adc_ready) begin
            if (adc_request) begin
            #1 //delay so the pulse doesn't disappear on the echo. TO BE REMOVED
          
            //manage trigger_value
            //... the trigger code is here

      //store data and move pointers around
      ring_buffer[++write_ptr] = adc_data;
      read_ptr++;

      adc_request = 0; //pull request down
    end
  end
\end{lstlisting}

\begin{lstlisting}[language=Verilog, caption={Code for requesting data from the ADC on posedge of the clock when in RUNNING state}]
      `RUNNING: begin
        timer++;
        if (~ adc_request)
          adc_request = 1; //request new adc value (handled with posedge adc_ready)
      end
\end{lstlisting}














